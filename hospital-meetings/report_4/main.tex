\documentclass[a4paper, 11pt]{article}
\usepackage{comment} % enables the use of multi-line comments (\ifx \fi) 
\usepackage{lipsum} %This package just generates Lorem Ipsum filler text. 
\usepackage{fullpage} % changes the margin

\begin{document}
%Header-Make sure you update this information!!!!
\noindent
\large\textbf{Doctor Jo\~{a}o Louren\c{c}o First Meeting} \hfill \textbf{Francisco Maria Calisto} \\
\normalsize ISR \& INESC-ID \\
Prof. Dr. Jacinto Nascimento \hfill R\&D Date: 24/02/2017 \\
Prof. Dr. Daniel Gon\c{c}alves \hfill Due Date: 24/02/2017

\section*{Report}

On February 24, 2017 (Friday) I went to the S\~{a}o Jos\'{e} in Lisbon at 09:50 for a meeting with Doctor Jo\~{a}o Louren\c{c}o at 10:00 in the morning as agreed by e-mail.

When I arrived at the Hospital, I encountered an organizational problem, because it was difficult for the information about my arrival to be properly communicated to the Doctor, and I had to send an SMS later to the doctor. In the answer to the SMS, the Doctor said to meet in the meeting room of Radiology so I immediately asked in the reception of the same area where this room was. They gave me the incomplete information so I had to ask another person in the hall. Arriving at the Doctor already with a delay of 15min due to all this, he asked me to sit down and had the remaining 45min to deal with a problem on the phone.

I took advantage of this waiting time by using it to understand the surrounding environment where the doctor works. In this case, there was a large room with a meeting table in the middle and against the wall there were 6 monitors radiological diagnosis. The material seemed good and in good conditions, unlike the Hospital's infrastructure. I remember the monitors being Philips.

At approximately 11:00, I then began to approach and get to know the doctor and future user of our interface, as well as the future collaborator of the project. Dr. Jo\~{a}o Louren\c{c}o \cite{joaoLourenco} specializes in Radiology and currently works at both Hospital São José and CUF Clinic in Belém and Hospital CUF in Torres Vedras. Essentially, and in an area of ​​interest as a user, it uses a lot of RIS software to obtain patient information and PACS, in this case the Philips eSight program, for the radiologic diagnosis. The features you use most are annotations written on the image and change the various views (windows) of the screen. It also uses the functionality of "Reformations" that will have to be later analyzed by us, the researchers.

In explaining the subject was somewhat reluctant to the fact that the project in its interpretation does not bring any sense of scientific innovation. So I explained that a new form of diagnosis would be implemented with a machine-readable approach to clinicians. It was mentioned in the possible existence of two types of software already on the market, Computer-Aided Detection (CADe) and Computer-Aided Diagnosis (CADx) \cite{computerAidedDiagnosis}. The former has an integrated implementation with machine learning so we have to keep this in mind in the future.

% to comment sections out, use the command \ifx and \fi. Use this technique when writing your pre lab. For example, to comment something out I would do:
%  \ifx
%	\begin{itemize}
%		\item item1
%		\item item2
%	\end{itemize}	
%  \fi

%\begin{thebibliography}{9}
%\bibitem{Robotics} Fred G. Martin \emph{Robotics Explorations: A Hands-On Introduction to Engineering}. New Jersey: Prentice Hall.
%\bibitem{Flueck}  Flueck, Alexander J. 2005. \emph{ECE 100}[online]. Chicago: Illinois Institute of Technology, Electrical and Computer Engineering Department, 2005 [cited 30
%August 2005]. Available from World Wide Web: (http://www.ece.iit.edu/~flueck/ece100).
%\end{thebibliography}

\begin{thebibliography}{9}
\bibitem{joaoLourenco}  Dr. Jo\~{a}o Louren\c{c}o. \emph{CUF Bel\'{e}m Cl\'{i}nica}. Lisbon, Portugal (EU). Available from World Wide Web: (https://www.saudecuf.pt/belem/encontre-um-medico/joao-lourenco).
\bibitem{computerAidedDiagnosis}  Computer-aided diagnosis. \emph{Wikipedia}. Available from World Wide Web: (https://en.wikipedia.org/wiki/Computer-aided\_diagnosis).
\end{thebibliography}

\end{document}
