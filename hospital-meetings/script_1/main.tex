\documentclass[a4paper, 11pt]{article}
\usepackage{comment} % enables the use of multi-line comments (\ifx \fi) 
\usepackage{lipsum} %This package just generates Lorem Ipsum filler text. 
\usepackage{fullpage} % changes the margin

\begin{document}
%Header-Make sure you update this information!!!!
\noindent
\large\textbf{High-Fi Prototype Right Click User Test Script} \hfill \textbf{Francisco Maria Calisto} \\
\normalsize ISR \& INESC-ID \\
Prof. Dr. Jacinto Nascimento \hfill R\&D Date: 20/02/2017 \\
Prof. Dr. Daniel Gon\c{c}alves \hfill Due Date: 24/02/2017

\section*{Simple Annotation}

Follow the next steps:

1) Access to the demo link from the project page:

https://mimbcd-ui.github.io/

2) Observe the user interface;

3) Tell us how much do you like or dislike the user interface feeling;

4) Right click on the image;

5) Annotate some image source;

\section*{Undo}

Follow the next steps:

1) Access to the demo link from the project page:

https://mimbcd-ui.github.io/

2) Observe the user interface;

3) Tell us how much do you like or dislike the user interface feeling;

4) Right click on the image;

5) Annotate some image source;

6) Undo the last annotation at the image source;

\section*{Redo}

Follow the next steps:

1) Access to the demo link from the project page:

https://mimbcd-ui.github.io/

2) Observe the user interface;

3) Tell us how much do you like or dislike the user interface feeling;

4) Right click on the image;

5) Annotate some image source;

6) Redo the last undone annotation at the image source;

\section*{Erase}

Follow the next steps:

1) Access to the demo link from the project page:

https://mimbcd-ui.github.io/

2) Observe the user interface;

3) Tell us how much do you like or dislike the user interface feeling;

4) Right click on the image;

5) Annotate some image source;

6) Erase the last annotation at the image source;

\section*{Text Annotation}

Follow the next steps:

1) Access to the demo link from the project page:

https://mimbcd-ui.github.io/

2) Observe the user interface;

3) Tell us how much do you like or dislike the user interface feeling;

4) Right click on the image;

5) Annotate some text field;

% to comment sections out, use the command \ifx and \fi. Use this technique when writing your pre lab. For example, to comment something out I would do:
%  \ifx
%	\begin{itemize}
%		\item item1
%		\item item2
%	\end{itemize}	
%  \fi

%\begin{thebibliography}{9}
%\bibitem{Robotics} Fred G. Martin \emph{Robotics Explorations: A Hands-On Introduction to Engineering}. New Jersey: Prentice Hall.
%\bibitem{Flueck}  Flueck, Alexander J. 2005. \emph{ECE 100}[online]. Chicago: Illinois Institute of Technology, Electrical and Computer Engineering Department, 2005 [cited 30
%August 2005]. Available from World Wide Web: (http://www.ece.iit.edu/~flueck/ece100).
%\end{thebibliography}

\end{document}
